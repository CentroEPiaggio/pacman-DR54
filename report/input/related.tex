\subsection{State of the art}
\label{sec:SoA}

One of the first early attempts to exploit active tactile exploration with passive stereo vision for object recognition was proposed in \citet{Allen1987Robotic}. In this paper, a rigid finger-like tactile sensor was used to trace along the surface with predefined movement cycles and provided a limited amount of information on object surface. The work was later extended to develop different exploratory procedures to acquire and interpret 3D touch information \citet{Allen1990Acquisition}. The exploratory procedures were, however, commanded by a human experimenter and therefore not linked to a fully autonomous system.
 
Single finger tactile exploration strategies for recognizing polyhedral objects have also been presented and evaluated in simulation, see \citet{Roberts1990ICRA} and \citet{Caselli1996ICRA}. In \citet{Moll2003STAR} a method for reconstructing shape and motion of an unknown convex object using three sensing fingers is presented. In this approach, friction properties must be known in advance and the surface is required to be smooth, i.e., it must have no corners or edges. Moreover, multiple simultaneous sensor contacts points are required resulting in additional geometric constraints for the setup.
 
%The early work by \citet{Allen1987Robotic} presents a hierarchical representation of the object. The tactile exploration strategy to refine is driven by local geometry features. This is engaged by using surface tracing algorithms. In that work, it is literally said: ``Given a starting and ending point on a surface, the sensor traces along the surface reporting its contact positions and normals as it moves along.'' However, no indication is given in how to determine those starting and ending points on the surface. \citet{Allen1990Acquisition} blah blah

In \citet{Petrovskaya2011Global} exploratory procedures have been considered with the aim to globally localize an object of known shape. Since the Bayesian posterior estimation for objects in 6D is known to be computationally expensive, this paper proposes an efficient approach, termed Scaling Series, that approximates the posterior by particles. For fully constraining datasets, this approach performs the estimation in under 1 s with very high reliability.

In \citet{Meier2011Probabilistic} the tactile shape reconstruction employs a Kalman filter, while in \citet{Bierbaum2008Potential} the tactile exploration is guided by Dynamic Potential Fields for motion guidance
of the fingers. Here, the authors show that grasp affordances may
be generated from geometric features extracted from the contact point set extracted during tactile exploration.

Interestingly, in \citet{Sommer2014Bimanual} a bimanual compliant tactile exploration is addressed that uses the GP representation to smooth noisy point data, but does not exploit the GP representation to define specific exploratory strategies.

\cite{Dragiev2011Gaussian} is one of the first works that employs Gaussian Process Implicit Surfaces (GPIS) for the concurrent representation of the object shape and to guide grasping actions towards the object. However, this work concentrates only on the mean of the shape distribution, i.e. the maximum a posteriori (MAP) estimate of the shape and practically ignores one of the gains of the GP --- the error bars.
Later work by the same authors in \cite{Dragiev2013Uncertainty} offers also a way to give preference to regions of the model with particular certainty level and introduce the notion of explore-grasp and exploit-grasp primitives.

\citet{Bjorkman2013Enhancing} focus the attention on building object models that can be extracted with a small number of actions (touches) with the ultimate aim of understanding the category objects belong to, rather than exhaustively trying to explore the whole object. In this paper, the implicit function representation of the object surface is modelled by Gaussian Process regression, where the shape of the GP is governed by a thin plate covariance function derived by \citet{Williams2007Gaussian}. A set of predefined tactile glances are performed on the object: however, these are not updated as the object model gets refined as successive touches are performed.

This paper is one of closest work to ours among the references. The main difference being: the space in which the next-best exploratory action is computed, the terminal condition for the overall algorithm, and the descriptor used. \citet{Bjorkman2013Enhancing} use Zernike moments which imply extra computation time, and lack of a probabilistic interpretation, which is one of advantages of using Gaussian Process in first place. Moreover, the exploratory actions are searched in a discrete space in the vertical direction and the approach angle, which seem extrinsic to the shape model. Using the ambient space instead of the intrinsic representation does not guarantee that the new observations will be on the desired shape region to be explored.  Finally, the number of actions, or touches in this case, are limited to a certain number, and then ordered according to the closes point on the implicit function with higher variance. In contrast, we set the highest expected variance in the shape prediction, so we explore until, probabilistically speaking, that goal is achieved.

\subsection{Background}
\label{sec:background}
\todo[inline,author=Federico]{Sorry about the silly question, what is the difference between this section and the previous ?}

% There are two broad directions to represent object shapes. Use either an implicit or an explicit (a.k.a. parametric) representation. A good comparison between the two is provided by \citet{Pirri2006About}.

Regarding the shape representation...

\citet{Faria2010Probabilistic} probabilistic representations of shapes

Using Gaussian Process for shape modelling is having an intensive development in recent years \citep{Mahler2015Grasp,Rosales2014Active,Bjorkman2013Enhancing,Dragiev2011Gaussian}, due to its versatility to accomadate noisy information, provide smooth regression of data, and a natural way to fuse different sources of information \citep{Rasmussen2006Gaussian}.

Manifold gaussian process: \citet{Calandra2014Manifold} ``The quality of a Gaussian Process model strongly depends on an appropriate covariance function.''

The tree is constructed incrementally from samples drawn randomly from the search space and is inherently biased to grow towards large unsearched areas of the problem
Space-filling trees are
Rapidly-exploring random trees \citet{LaValle2011Motion}

Continuation method for implicitly defined surfaces \citet{Henderson1993COMPUTING}

More general continuation method used in a more complex scenario combined with rapidly-exploring random trees \citep{LaValle2011Motion} by \citet{Jaillet2013Path}, where the idea is to explore the part of the manifold defined by several kinematic loop constraints that solves the motion planning query.

\citet{Zhu2009Nonrigid} succesfully recover non-rigid shape, means that gaussian processes can be used

\citet{Dragiev2011Gaussian} uses a squared-exponential-like covariance function. This is selection seems to be in agreement with the use is given to the shape estimation as a gradient field that drives the reach-to-grasp controller in a smooth way.

\citet{Rosales2014Active} exploits the Gaussian Process modelling the object shape to find geodesic trajectories on the surface that are later followed by an exploratory probe to gather frictional properties. However, no further use of that propery is given.

\citet{Mahler2015Grasp} proposes similar ideas as those given in \citet{Dragiev2011Gaussian}, in fact they follow the same covariance function for the shape modelling, but adds a local optimization step to have a more elaborated grasp controller that drives the hand to the most-likely succesful object grasp using a metric that involves the well-known Ferrari-Canny measure.

\citet{Williams2007Gaussian} derives the thin-plate covariance function for 3D shapes represented by implicit functions. The property of the thin-plate function to keep the tendency outside the training data is ideal for tactile exploration~\citep[Fig.~2]{Williams2007Gaussian}

Still not sure whether it should go in intro, soa, or background.
\citet{Li2016Dexterous} 
