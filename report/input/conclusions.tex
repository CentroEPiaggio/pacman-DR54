Furnish a robot with the ability to model and explore novel and arbitrary shapes seems adequate in the classical and new-coming robotic applications such as object grasping and manipulation or tree inspection by drones. 
In this work, we gathered concepts already proven to be successful, namely Gaussian process implicit surfaces and the determination of implicitly-defined manifolds via continuation techniques to derive a next-best candidate strategy for shape exploration exploiting the probabilistic interpretation of Gaussian processes. 
The strategy returns paths that increases the model accuracy in terms of probabilistic variance of the predicted shape. The novelty with respect to previous approaches is twofold.
First, the method does not require the computation of the explicit form of the predicted shape. As a matter of fact, that would contradict the choice of Gaussian process implicit surfaces as a shape representation in the first place. 
%In our work, the purpose of this aspect is due to the exploration by a real probe, such as our Vito robot with its intrinsic tactile sensor in one arm, and the idea of collision avoidance using an unknown object is actually an interesting topic in our opinion. 
Second, the strategy makes no assumptions on the exploratory probe as the strategy is intrinsic to the representation. This property allowed us to compare the strategy in a synthetic scenario, where the probing was done using a basic ray-cast on the ground truth of the shape. The proposed strategy was compared to a naive one, where the rays were directed randomly. Our strategy outperforms in a proportion of X using a poking action and in a proportion of Y using a sliding action.
The strategy was also tested successfully using our Vito robot as the exploratory probe.

Several points deserve further attention. Perhaps, the most relevant to this work is the consideration of gradient observations as described by \citet{Solak2003Derivative}, specially due to our hardware setup. This feature has been presented in related works but do not seem exploit it. Another interesting point arises when discussing how to locally explore the model, that is, the direction to go within a Chart. 
A final interesting topic is the use of the proposed strategy to drive control loop, where the controller command can be a partial, or even the first  single step, of the next-best exploratory path. According to our experience, and to the high speed a path over a novel shape is computed, we believe this is feasible, at the same time as a promising road to make.
% The incorporation of other exploration primitives is also an interesting topic, that is, in our case we returned a path, but also dynamic aspects in the form of time-parametrized path (a trajectory) can yield other kind of information that can leave to a better model.