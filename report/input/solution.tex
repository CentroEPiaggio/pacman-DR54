First, we provide a pre-processing step with the objective of having a segmented point set of the object \ref{sec:pre}. This is then used to create the probabilistic shape representation \ref{sec:object}. This is then exploited in 

%%%%%%%%%%%%%%%%%%%%%%%%%%%%%%%%%%%%%%%%
\subsection{Pre-processing}
\label{sec:pre}

Given a point cloud of the current scene, say an object on a table, the ubiquituos tabletop object segmentation 
\citep{TabletopObjectDetector} can be used to have 

%%%%%%%%%%%%%%%%%%%%%%%%%%%%%%%%%%%%%%%%
\subsection{Shape modelling}
\label{sec:object}
What makes a good surface representation?
\begin{itemize}
\item Accurate (we handle this with probability)
\item Concise (we might want this)
\item Intuitive specification (we don't need this)
\item Local support
\item Affine invariant 
\item Arbitrary topology (we need this)
\item Guaranteed continuity (we need this)
\item Natural parameterization (we don't need this)
\item Efficient display (we shouldn't need this)
\item Efficient intersections (we could need this)
\end{itemize}

How do we define implicit function?
\begin{itemize}
\item Algebraics
\item Blobby models
\item Skeletons
\item Procedural
\item Samples
\item Variational
\item Gaussian Process !
\end{itemize}

Variational surfaces:
\begin{itemize}
\item Advantages:
\begin{itemize} 
\item Easy to test if point is on surface
\item Easy to compute intersections/unions/differences
\item Easy to handle topological changes
\end{itemize}
\item Disadvantages:
\begin{itemize}
\item Indirect specification of surface
\item Hard to describe sharp features
\item Hard to enumerate points on surface !
\item Slow rendering !
\end{itemize}

\begin{algorithm}[h]
\textbf{\textsc{FitGaussianProcess}}$(P, params)$\\ %functionname
\LinesNumbered
\DontPrintSemicolon
\SetAlgoVlined \SetKwInOut{Input}{input} \SetKwInOut{Output}{output}
\Input{The segmented object shape in form of a point cloud, $P$, and the
to be used for the Gaussian Process fitting.}
\Output{The shape model}
  Generate inside and outside points \\
  Merge and label data into the training data $X$ \\
  $\mathcal{GP} \leftarrow$ \textsc{InitModel}$(params)$ \\
  $\mathcal{GP} \leftarrow$ \textsc{FitToData}$(X)$ \\
  \Return $\mathcal{GP}$ \\
\caption{The object shape model generation} \label{algo:strategy}
\end{algorithm}

%%%%%%%%%%%%%%%%%%%%%%%%%%%%%%%%%%%%%%%%
\subsection{Explorattion strategy}
\label{sec:strategy}


\begin{algorithm}[h]
\textbf{\textsc{GPAtlasRRT}}$(\mathcal{GP}, V_{max}, type)$\\ %functionname
\LinesNumbered
\DontPrintSemicolon
\SetAlgoVlined \SetKwInOut{Input}{input} \SetKwInOut{Output}{output}
\Input{The GP The action $type$ can have for now two values, $poke$ for a single
touch, $slide$ for path. This depends on the available hardware.}
\Output{The best next point if any, otherwise null.}
  $e \leftarrow 1$ \\
  \While { $e < h$ } 
  {
    $i \leftarrow 0$ \\
    \While{ $i<n$ }
    {
      \Return action 
    }
    $e \leftarrow e + 1$ \\
  }
  \Return{\sc Null}

\caption{The best-next action planner} \label{algo:strategy}
\end{algorithm}

%%%%%%%%%%%%%%%%%%%%%%%%%%%%%%%%%%%%%%%%
\subsection{Solution in a nutshell}
\label{sec:summary}

\begin{algorithm}[h]
\textbf{\textsc{ObjectShapeExploration}}$(P, V_{max}, type, params)$\\ %functionname
\LinesNumbered
\DontPrintSemicolon
\SetAlgoVlined \SetKwInOut{Input}{input} \SetKwInOut{Output}{output}
\Input{Point cloud of the segmented object}
\Output{The object model predicting the shape up to the specified value $V_{max}$}
  PreProcessing \\
  $\mathcal{GP} \leftarrow $\textsc{FitGaussianProcess}$(P, params)$\\
  \While { \texttt{true} } 
  {
    $a \leftarrow $\textsc{GPAtlasRRT}$(\mathcal{GP}, V_{max}, type)$ \\
    \eIf{ $a$ = \sc Null }
    {
      \Return {$\mathcal{GP}$} \\
    }
    {
      \textsc{ApproachTo}($a$) \\
      \textsc{Execute}($a$) \\
    }
  }
  \Return 

\caption{Best-next action method} \label{algo:strategy}
\end{algorithm}

In the \textsc{ApproachTo} phase, the robot moves using position control and collision avoidance. In the \textsc{Execute} phase, the robot moves using Cartesian impedance control, with the Cartesian force, pose and impedance set properly for the given setup. Implementation details of the solution depicted in Algorithm~\ref{alg:strategy} are given in the next section.
