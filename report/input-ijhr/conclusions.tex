This paper presented a combined shape representation and planning algorithm (GPAtlasRRT). Together they allow planning of local tactile exploration of an incompletely modelled object. We demonstrated the benefits both in simulation, and on a real robot. The real robot system also demonstrated the ability to grasp an object with one hand, segment this hand from the object in an initial point cloud, and then extend the model with touches guided by GPAtlasRRT.

This local tactile planning strategy required bringing together Gaussian process implicit surfaces and the determination of implicitly-defined manifolds via continuation techniques. This exploited the ability of Gaussian processes to naturally represent model uncertainty. 

The beneficial features of the approach are several. First, the planning method does not require the computation of the explicit form of the entire predicted shape. Second, the strategy makes no assumptions about the exploratory probe. Third, it can plan sequences of tactile actions across a contiguous portion of the object surface, thus providing a detailed surface reconstruction. Fourth, the robot implementation allows the robot to explore an object as it holds it.

The proposed strategy was compared to a naive one, where touch rays were directed randomly. Our strategy outperforms this, whether using a single touch, or a touch sequence. The strategy was also tested successfully using our Vito robot as the exploratory probe.

Several points deserve further attention. Perhaps most relevant to this work is the consideration of gradient observations as described by \cite{Solak2003Derivative}, especially due to our hardware setup. This feature has been presented in related work but not previously exploited. Another interesting point arises when discussing how to locally explore the model, that is, the direction to move within a chart. 
A final interesting topic is the use of the proposed strategy to drive a control loop where the controller command can be part of, or even just the first single step, of the exploratory path. According to our experience this is feasible and promising road to explore.
% The incorporation of other exploration primitives is also an interesting topic, that is, in our case we returned a path, but also dynamic aspects in the form of time-parametrized path (a trajectory) can yield other kind of information that can leave to a better model.